\documentclass[12pt,a4paper]{article}

\usepackage[utf8]{inputenc}
\usepackage{gensymb}
\usepackage{amsmath}

\begin{document}

\title{Introduction To Aerospace}
\date{}
\maketitle

\newpage

\tableofcontents

\newpage

\section{Introduction}
	
	This module aims to introduce the basic concepts of Aerospace Engineering.

	\subsection{Assessment}
		There are two sets of coursework given out, each worth 12.5\%. There is also a final exam in January worth 75\%.
		
	\subsection{Office Hour}
		You are able to drop in, you do not need to make an appointment. Office hours are every Tuesday 11-12 in A204 EC.
		
	\subsection{Reading}
		A book that contains much more information than you will need this year, but is still useful, is:
		\begin{itemize}
			\item Introduction to Flight by John D. Anderson Jr.
		\end{itemize}
		
\newpage

\section{The Atmosphere}
	The atmosphere is constantly changing with varying temperature and pressure.
	
	\subsection{The International Standard Atmosphere (ISA)}
		A standard atmosphere was created so that aircraft and aircraft engines could be compared under the same conditions. \\
		The ISA represents average conditions in North America and Europe and is based on the assumptions that:
		\begin{itemize}
			\item Air is a perfect gas
			\item Air is dry
			\item Gravity doesn't change with altitude
			\item That hydrostatic equilibrium exists
		\end{itemize}
		These assumptions are all innacurate but are close enough.
		
	\subsection{Regions}
		The atmosphere is divided into four regions.
		\begin{itemize}
			\item Troposphere
			\begin{itemize}
				\item Around 11km in height
			\end{itemize}
			\item Stratosphere
			\begin{itemize}
				\item 11km to 53km
			\end{itemize}
			\item Mesosphere
			\begin{itemize}
				\item 53km to 105km
			\end{itemize}
			\item Thermosphere
			\begin{itemize}
				\item Up to 400km
			\end{itemize}
		\end{itemize}

	\subsection{Equation of State}
		\[p=\rho RT\]
		For a perfect gas R is constant, therefore:
		\[\frac{p}{p_{sls}}=\frac{\rho}{\rho_{sls}}\frac{T}{T_{sls}}\]
		So, the above equation can be written as 
		\[\delta = \sigma \theta \]
		
	\subsection{Lapse Ratios}
		\subsubsection{Temperature}
			Within the troposphere there is a temperature lapse rate of -6.5\degree C per km, meaning for every kilometer you go above sea level the temperature drops by 6.5\degree C. 

			\begin{align*}
				T &= 288.15-6.5*10^{-3}h \\
				\theta &= 1-2.2558*10^{-5}h 
				\intertext{Where:}
				T &= \text{Temperature (\degree K} \\
				\theta &= \text{Temperature Ratio} \\
				h &= \text{Height (m)}
			\end{align*}						
			
		Outside of the troposphere the Outside Air Temperature (OAT) is -56.5\degree C and remains at this temperature until around 25km.
		\begin{align*}
			T &= 216.65 \\
			\theta &= 0.7519
		\end{align*}
		
		\subsubsection{Pressure}
			The variation of pressure is determined by combining the Equation of State and the intergration of the Hydrostatic Equation.
			\[ \delta p=-\rho gh\]
			
			Within the troposphere:
			\[ \delta =(1-2.2558*10^{-5}h)^{5.25588} \]
			
			Within the lower stratosphere:
			\[\delta =0.22336e^{1.7346-1.5769*10^{-4}h} \]
			
		\subsubsection{Density}
			We can combine the Equation of state and the equations for temperature and pressure lapse ratio to give
			\begin{align*}
			\sigma &= \frac{\delta}{\theta} \\
			\sigma &= \frac{-\rho gh}{1-2.2558*10^{-5}h}
			\end{align*}
			
			
	\subsection{The Speed of Sound}
		The speed of sound can also be known as the local sonic velocity, which is defined as "the speed at which a small disturbance propagates through air", varies according to the temperature lapse ratio.
		\begin{align*}
			a&=a_{sls}\sqrt{\theta}
			\intertext{Where:}
			a&=\text{Speed of Sound} \\
			a_{sls}&=\text{Speed of Sound at Sea Level}
		\end{align*}
		At sea level the speed of sound is 340.3 m/s.
		\subsubsection{Mach Number}
			The mach number can be found using
			\begin{align*}
				M&=\frac{V}{a}
				\intertext{Where:}
				M &=\text{Mach Number} \\
				V&=\text{Local Speed of the Object} \\
				a&=\text{Local speed of sound}
			\end{align*}
			The local speed is also known as the True Air Speed (TAS).
			
	\subsection{Reynolds Number}
		The reynolds number is dimensionless and can help us predict flow patterns. A low reynolds number means that the flow tends to be laminar (stable), while a high reynolds number means that the flow is turbulent (unstable). A turbulent flow can form eddy currents which churn up the flow, using energy and increasing the chances of cavitation. \\
		The reynolds number is usually defined as 
		\begin{align*}
			Re&=\frac{\rho LV}{\mu}=\frac{LV}{v}
			\intertext{Where:}
			L&=\text{Length} \\
			V&= \text{Velocity} \\
			\mu&=\text{Dynamic Viscosity} \\
			v&=\text{Kinematic Viscosity}
		\end{align*}
			
\end{document}