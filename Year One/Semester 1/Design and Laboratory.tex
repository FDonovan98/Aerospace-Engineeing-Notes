\documentclass[12pt,a4paper]{article}

\usepackage[utf8]{inputenc}


\begin{document}

\title{Design and Laboratory}
\date{}
\maketitle

\newpage

\tableofcontents

\newpage

\section{Introduction}
	Office hours are Tuesday 12-13 in A128 Engineering East, you will need to make an appointment via email first. 

	\subsection{Assessment}
		50\% is split between two solidworks assignments; one in October worth 14\% and another in November worth 36\%. \\
		The other 50\% is based on lab worw in december. 25\% will be a Materials Selection assignment, 10\% a Mechanical Testing assignment, and the last 15\% a Fluid Laboratories assignment. 
		
\newpage
		
\section{Engineering Drawings}
	A 3D view is useful for visualisation but is not enough on it's own as it is not useful for anything else. \\
	\\
	An engineering drawing is the best way to communicate engineering ideas as they allow accurate visualisation of components or assembely, and provide details required for manufacture. A drawing forms a \textbf{legal} contract between the designer and the manufacturer. \\
	\\
	Drawings should always follow International Standards, whether thats ISO, BSI, or ANSI. 
		
	\subsection{Basics of a Drawing}
		\subsubsection{Orthographic Projection}
			An engineering drawing should have at least a view from the front and side. \\
			\\
			In Europe and Asia the first angle view is used, where you draw a front view and then the view of the left side. \\
			In the UK and USA third angle view is used, where you draw a front view and then the view of the right side.
			
		\subsubsection{Lines}
			\begin{itemize}
				\item[Visible Lines -] Continuous and thick
				\item[Hidden Lines -] Dashed and thin
				\item[Centre Mark -] Mark circle or arc centres with a thin cross and thin dot-dash lines
				\item[Centre Lines -] Mark cylindrical features and lines of symmetry with a thin dash-dot line
			\end{itemize}
			
		\subsubsection{Dimensions}
			The drawing should be fully dimensioned, there should be no missing measurements but also nothing should be labeled twice! You should use thin lines and clear text. Units are not needed if they are stated elsewhere.
	
	\subsection{The Software}
		We will be using Solidworks for all engineering drawings. It is more important that the drawings are up to engineering standards rather than looking good.
	
	
\end{document}