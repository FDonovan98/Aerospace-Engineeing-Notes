\documentclass[12pt,a4paper]{article}

\usepackage{amsmath}

\begin{document}

\title{Strength Of Materials}
\date{}
\maketitle

\newpage

\tableofcontents

\newpage

\section{Introduction}
    The office hour is Tuesday 14:00 to 15:00 in C217 Engineering Central. \\
    \\
    This module is the study of how materials can sustain external actions without failure by using simplified mathematical models. Types of actions are:
    \begin{itemize}
        \item Forces
        \item Temperatures changes
        \item Settlements
    \end{itemize}
    While failure types include:
    \begin{itemize}
        \item Rupture
        \item Excessive deformation
    \end{itemize}

    \subsection{Assessment}
        80\% is a closed book exam at the end of the term. It is a core module, so must acieve at least 40\% in the exam. //
        20\% is based off of three blackboard tests; basic concepts (7\%), Basic beam theory (7\%), Stresses and strains and advanced beam theory (6\%). \\
        You will have one week exactly for each test, they will go live at 16:30 on a friday.

    \subsection{Recommended Reading}
        \begin{itemize}
            \item Hibbeler, RC, Mechanics of Materials, Prentice Hall, SI Eighth Edition
            \item Case, Chilver \& Ross, “Strength of Materials \& Structures”, 4th Edition
            \item D. Gross, W. Hauger, J. Shroder, W. Wall, J. Bonet, Engineering Mechanics 2: Mechanics of Materials, Springer
        \end{itemize}

\newpage

\section{Basic Concepts}
    \subsection{Governing Principles}
        How a load is transmitted through a material is goverened by two basic principles. 
        \subsubsection{Equilibrium}
            The sum of all forces and moments on a body, or any part of a body, must sum to zero. If a problem can be solved using only equilibrium conditions then it is statically determinate.

        \subsubsection{Compatibiblity}
            The movements resulting from the external loads must be internally compatible (the material must not break from movement caused by external loads) when also considering the external supports.

    \subsection{St. Venant's Principle}
        Regardless of the complexity of the distribution of external forces at a small region on the surface of a body, the resulting effect a small distance away depends only on the statically equivalent force. \\
        Results in stress concentrations in a material, rather than uniform stress distribution. \\

    \subsection{Stress}
        Stress is the amount of internal force per unit area:
        \[\sigma = \frac{F}{A}\]
        and can be either tensile or compressive. \\
        \\
        The force acting on an area can be normal or tangentail to the area. The direct stress ($\sigma"$) is the normal force per unit area, while shear stress ($\tau$) is the tangential force per unit area.

    \subsection{Temperature Stresses}
    \begin{align}
        \varepsilon_T&=\alpha \Delta T \\
        \Delta L_T&= \alpha \Delta TL \\
        \intertext{Where: }
        \varepsilon_T&=\text{Thermal strain} \nonumber\\
        \alpha &= \text{Thermal coefficient} \nonumber\\
        T&=\text{Temperature} \nonumber\\
        \Delta L_T &=\text{Change in length due to thermal expansion} \nonumber\\
        L&=\text{Original length} \nonumber
    \end{align}

    Thermal stress doesn't cause any strain, unless the expansion is restricted by a support or constraint. The general procedure for solving problems of thermal stress is: 
    \begin{enumerate}
        \item Remove one of the constrainst and assume the body can extend or contract freely
        \item Calculate $\Delta L_T$
        \item Calculate the force required to return the body to its original length
    \end{enumerate}

    \subsection{Force Stress}

    \begin{align}
        \Delta L_F&=\frac{FL}{AE} \\
        \intertext{Or}
        \Delta L_F&=\sigma \frac{L}{E}
        \intertext{Where: }
        \Delta L_F&=\text{Change in length due to force applied} \nonumber\\
        \sigma &=\text{Stress} \nonumber\\
        F&=\text{Force} \nonumber\\
        L&=\text{Original length} \nonumber \\
        A&=\text{Cross sectional area} \nonumber\\
        E&=\text{Young's modulus} \nonumber
    \end{align}

    \subsection{Poisson's Ration}
        When there is a direct stress along a beam in one direction this results in a direct strain in the same direction as well as a lateral strain, so a strain acting perpendicular to the direction of stress. The relationship between direct and lateral strain is given by Poisson's coefficient ($v$), and is usually 0.3.

        \begin{align}
            v&=-\frac{\varepsilon_2}{\varepsilon_1}
            \intertext{Where:}
            v &=\text{Poisson's Ration} \nonumber\\
            \varepsilon_2&=\frac{\Delta d}{d} \nonumber\\
            \varepsilon_1&=\frac{\Delta l}{l} \nonumber
        \end{align}

    \subsection{Strain Energy}
        When a material is deformed the work done by external forces is stored as elastic strain energy within the material. 
        \[W=\int F\delta l = \int \sigma Al \delta \varepsilon=V\int \sigma \delta \varepsilon\]
        The strain energy per unit volume ($w$) is the area under the stress-strain graph
        \[w=\int \sigma \varepsilon \delta \varepsilon\]

\end{document}