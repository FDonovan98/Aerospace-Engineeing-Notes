\documentclass[12pt,a4paper]{article}


\begin{document}

\title{Fluid Mechanics}
\date{}
\maketitle

\newpage

\tableofcontents

\newpage

\section{Introduction}
    This module deals with using and understanding the conservation laws of mass, energy and momentum. \\
    Lectures are recorded and will be uploaded to blackboard, along with slides, solved problems, and visualiser notes. \\
    \\
    Office hours will be on Tuesday 10:00-12:00 in A122 Engineering Central. It is recommended to come to this if you have a problem as he is unlikely to reply to emails. 

    \subsection{Assessment}
        100\% of the marks are from a final exam at the end of the year. In this exam:
        \begin{itemize}
            \item 35\% is from multiple choice questions
            \item 25\% is from a problem similar to those seen in class or in the excersises. This is again a multiple choice question with negative marks if you get the wrong answer
            \item 40\% will be from an aditional question based on the taught material. This will be a standard question (e.g not multiple choice)
        \end{itemize}

    \subsection{Recommended Reading}
        \begin{itemize}
            \item Fluid Mechanics: Fundamentals and Applications - YunusA. Çengel, John M. Cimbala, McGraw Hill
            \item Applied Fluid Mechanics: Global Edition - Robert L. Mott, Joseph A. Untener, Pearsons
        \end{itemize}

    \newpage

    \section{Basic Characteristics of Fluids and Flows}
        \subsection{The Continuum Assumption}
            Fluids are considered a continuum on length scales larger than the molecular ones. For example, you can't measure individual water 'grains' until you look at the molecular level.
        
        \subsection{Types of Problems}
            \begin{itemize}
                \item[Macroscopic - ] Deals with the overall flow (e.g amount of water entering and leaving a pipe)
                \item[Microscopic - ] Deals with a specific section of the flow (properties of the fluid flow within a small section of this pipe)
            \end{itemize}


\end{document}