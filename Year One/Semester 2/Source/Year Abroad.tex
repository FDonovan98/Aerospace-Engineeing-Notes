\documentclass[12pt,a4paper]{article}


\begin{document}

\title{Year Abroad}
\date{}
\maketitle

\newpage

\tableofcontents

\newpage

\section{Information}
    Must achieve at least 55\% first year and in the first semester of year 2. It extends the length of the degree by one year, meaning the marks from the year abroad aren't included in your degree calssification. You must however pass the year for it to be inclluded in your degree title. \\
    This year takes place after your second year. \\
    You must contact engoffice@swansea.ac.uk to enroll onto the scheme. \\
    The course you can choose to study abroad is not dependant on the course you are currently studying to a point. You are normally restricted to any STEM subject. \\
    Talk in first semester of second year with more details on costs and available destinations. \\
    You are expected to source your own accomodation, however partner universities do provide relevant information. \\
    If you go out of europe you cannot have any resits for second year.

\section{Cost}
    15\% of the standard tuition fee goes to swansea, you do not need to pay anything to the host university! You are still able to apply for student finance as usual. \\
    The university also provides funding. Last year offered £1000 a year if you go outside of europe. \\
    There are extra bursaries available if you go to a pertner university in Texas. \\
    Bursaries if you go into europe will depend on brexit negotiation (so no then). \\

\section{Alternatives}
    Summer programs are available if a year sounds like too much. \\
    Internships in china are available for between one and three months. \\
    Up to £800 is available for these programs, dependant on the length of the programme. 

\end{document}