\documentclass[12pt,a4paper]{article}

\usepackage{amsmath}

\begin{document}

\title{Thermodynamics 1}
\date{}
\maketitle

\newpage

\tableofcontents

\newpage

\section{Energy, Energy Transfer and General Energy Analysis}
    \subsection{Types of Energy}
        The sum of all energies in a system is called Total Energy ($E$). This includes the macroscopic and internal (microscopic) energy. \\
        Total energy per unit mass of a system is
        \[e=\frac{E}{m}\]

        Macroscopic forms of energy are those possessed by the system relative to an external frame of refrence, such as kinetic energy or potential energy. \\
        \\
        Internal or microscopic forms of energy is related to the molecular structure of the system. This includes the latent energy (phase change,), sensible energy (temperature change) and chemical or nuclear energy. \\
        \\
        Kinetic Energy (KE)
        \begin{align*}
            KE&=m\frac{V^2}{2} \\
            \intertext{Where:}
            m &= \text{mass}  \\
            V&=\text{Velocity} 
        \end{align*}

        Potential Energy (PE)
        \begin{align*}
            PE&=mgz \\
            \intertext{Where:}
            m&=\text{mass}  \\
            g&=\text{gravity}  \\
            z&=\text{height}  
        \end{align*}

        Flow Energy
        \begin{align*}
            FE &=m\frac{p}{\rho} \\
            \intertext{Where: }
            m&=\text{mass}  \\
            p&=\text{pressure}  \\
            \rho&=\text{density} 
        \end{align*}

    \subsection{mechanical Energy}
        This is a form of energy which can be converted directly and completely to mechanical work using an ideal (lossless) mechanical device. Kinetic and potential energies are both mechanical energies, while thermal energy would not be due to the second law of thermodynamics. \\
        \\
        The mechanical energy of a flowing fluid per unit mass can be calculated by summing the flow energy kinetic energy, and potential energy of the fluid. \\
        \\
        Volume Flow Rate ($\dot{V}$)
        \[\dot{V}=V_{avg}A_C\]
        \\
        Mass Flow Rate ($\dot{m}$)
        \[\dot{m}=\rho \dot{V}\]
        \\
        Energy Flow Rate ($\dot{E}$)
        \[\dot{E}=\dot{m}e\]
        \\
        Therefore, the flowrate of mechanical energy could be written as 
        \[\dot{E_{mech}}=\dot{m}e_{mech}\]
        This  assumes the fluid flow is incompressible, so that $\rho$ is constant. \\
        \\
        As energy can be neither created nor destroyed, if $\Delta e_{mech}>0$ then work has been supplied to the fluid, while if $\Delta e_{mech}<0$ then work has been extracted from the fluid.\\
        This means that the maximum power generated in an ideal system is 
        \[\dot{W_{max}}=\dot{m} \Delta e_{mech}\]
        \[\Delta e_{mech}=\frac{P_2-P_1}{\rho}+\frac{V^2_2-V^2_1}{2}+g(z_2-z_1)\]
        \\
        When $\Delta e_{mech} =0$ we know that it is a steady, incompressible flow. 

    \subsection{Bernoulli Equation}
        \[\frac{P}{\rho}+\frac{V^2}{2}+gz=\text{constant (along a streamline)}\]
        The Bernoulli equation is an approximate relation between pressure, velocity, and elevation for fluid particles along a streamline. It is only valid in regions of steady, incompressible flow where net frictional forces are negligible. \\
        \\
        This equation is usually in the form
        \[\frac{1}{2}\rho v^2 + \rho gh + P = constant\]
        So
        \[\frac{1}{2}\rho_1 v_1^2 + \rho_1 g_1h_1 + P_1 = \frac{1}{2}\rho_2 v_2^2 + \rho_2 g_2h_2 + P_2\]
        
\newpage

\section{Energy Analysis of a Closed Systems}
    \begin{itemize}
        \item[$C_v$ -] Specific heat capacity at a fixed volume
        \item[$C_p$ -] Specific heat capacity at a fixed pressure 
    \end{itemize}
    $C_p$ is always higher than $C_v$. \\

    \begin{align*}
        \Delta u&=C_v(T_1-T_2) \\
        \intertext{Where: }
        \Delta u&=\text{ Change in internal energy}  \\
        C_v&=\text{ Specific heat capacity with fixed volume}  \\
        T_1&=\text{ Initial temeprature} \\
        T_2&=\text{ Final temperature} 
    \end{align*} 

    \subsection{Energy Balance Equation in a Single Inlet Outlet System}
        \begin{align*}
            \dot{Q}-\dot{W}&=\dot{m}\left(h_2-h_1 + \frac{v_2^2-v_1^2}{2}+g(z_2-z_1)\right)
            \intertext{Where:}
            \dot{Q} &= \text{Change in thermal energy} \\
            \dot{W} &= \text{Work done} \\
            \dot{m} &= \text{Mass flow rate} \\
            h &= \text{Enthalpy energy} \\
            v &= \text{Velocity} \\
            g &= \text{Gravity} \\
            z &= \text{Height}
        \end{align*}
        
\newpage

\section{Energy and Mass Analysis of Steady Flow Control Volumes}
    Brayton cycle dals with external combustion in a gas turbine, such as a jet engine. When analysis a complex system it's best to break it up into its sections, such as the intake, compressor, combustion chamber, turbine and nozzle in a jet engine. 

    \subsection{Nozzles and Diffusers}
        A nozzle increases the velocity of the fluid by decreasing the pressure. A diffuser does the exact opposite, increasing the pressure by slowing the fluid down. \\
        \\
        In these systems we can assume there is no work done ($\dot{W}=0$), negligible change in potential energy ($\Delta pe = 0$), and can often assume there is negligible heat transfer ($\dot{Q} = 0$). \\
        This leaves us with only enthalpy and kinetic energy.
\end{document}