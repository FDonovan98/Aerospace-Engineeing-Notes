\documentclass[a4paper, 12pt]{article}

\usepackage{mhchem}
\usepackage{graphics}
\usepackage{graphicx}
\usepackage{chemfig}
\usepackage{tabularx}
\usepackage{amsmath}
\usepackage{multicol}
\usepackage{gensymb}

\begin{document}
\title{Foundatin Chemistry}
\date{}
\maketitle 

\tableofcontents

\newpage

\section*{Introduction}
	\addcontentsline{toc}{section}{Introduction}
	All lectures will take place within the Great Hall and all practicles shall take place in lab C18. This can be accesed from the left hand corridor of Engineering Central. All notes are available on the Blackboard.

	\subsection{Recommended Reading}
		\begin{itemize}
			\item Chemistry : molecules, matter, and change / Peter Atkins, Loretta Jones 
		\end{itemize}
	
	\subsection{Testing}
		There will be tests on Blackboard throughout the year which will make up 10\% of the final mark. There will also be two practicals, each worth 7.5\%, and a final end-of-year test worth 75\%

\newpage

\section{Structure and Bonding}
	\begin{description} 
		\item [Atomic number (Z) - ] The number of protons in an atom's nucleus
		\item [Atomic mass ($A_r$) - ] The number of protons and neutrons within an atomic nucleus
		\item [Isotopes - ] The same element (so the same Z) but with a different $A_r$, meaning more neutrons within the nucleus
		\item [Volatile - ] Evaporate at normal (room) temperatures and pressures
		\item [Semiconductor - ] A material with an electrical conductivity less than a conductor, such as copper, but greater than an insulator, such as glass. Their resistance decreases as temperature increases, which is the opposite of how metals behave.
	\end{description}

	The $A_r$ displayed on some Table of Elements is a decimal as the $A_r$ is the average of all isotopes of that element. For example, 99\% of naturally occuring carbon is \ce{^{12}C} and 1\% is \ce{^{13}C}, meaning that:
	\begin{align*}
		A_r(C) &= \frac{99}{100}(12) + \frac{1}{100}(13) \\
			  &= 12.011
	\end{align*}

	\subsection{Mass Spectrometer}
		A mass spectrometer is used to measure the atomic mass of an element. This is a five step process.
	\begin{enumerate}
		\item The sample is vapourised
		\item The vapour is bombarded by an electron beam, ionising it positively
		\item The ions are accelerated by an electric field. The strength of this field is known
		\item The ions are deflected by a magnetic field. The strength of this field is known
		\item Ions are detected by a plate
	\end{enumerate}
	
	Volatile samples are injected into the ionization chamber, while involatile samples are first vapourised with pre-heating. The mass spectrometer is held under vacuum. In the ionisation chamber when the electrons collide with the vapourised atoms the electron stream can knock electrons out of the outer shell.

	\[ \underbrace{e^-}_\text{high-energy electron} + \underbrace{X}_\text{Sample atom} \to \underbrace{X^+}_\text{Positive ion} + \underbrace{e^-}_\text{electron knocked out of X} + \underbrace{e^-}_\text{retreating high-energy electron} \]

	The positive then pass through holes in two charged plates, accelerating the ion forwards. These ions pass through the magnetic field generated by the electromagnet and are deflected dependant on their mass. Constant electric and magnetic fields are applied so only ions of a particular mass are deflected to the ion detector. Lighter ions are deflected the most while heavier ions are deflected more. Therefore gradually increasing the magnetic or electric field allows ions of different masses to hit the detector. \\
	Masses can be calculated from the known fields, but the instrument is normally calibrated with a known mass. The relative peak heights give the relative abundance of different ions present. 

	\begin{figure}[h]
		\includegraphics[width = \textwidth]{"Mass spectrometer".png}
	\end{figure}

	Molecules have more complex spectometry graphs than atoms as they can be made of different isotopes of the same atom. For example \ce{Cl_2} can be \ce{^{35}Cl-^{35}Cl}, \ce{^{35}Cl-^{37}Cl}, or \ce{^{37}Cl-^{37}Cl}.

	\subsection{Patterns in the periodic table}
	The simplest classification in the periodic table is to split it into metals, metalloids, and non-metals.
	\begin{description}
		\item[Metals - ] Metals are typically lustrous solids which conduct electricity and have high melting points. They have a gigantic metallic structure and often act as reducing agents in reactions.
		\item [Non-Metals - ] Non-metals are typically solids, liquids or gasses, that have a discrete molecular structure and do not conduct electricity. They generally have low melting points.
		\item [Metalloids - ] These have characteristics between metals and non metals, being reasonably strong solids with a giant molecular structure. They are neither conductors nor insulators and have very high melting points. These materials are semiconductors.
	\end{description}
	
	\subsubsection*{Groups}
	There is still a lot of variation within these groups so more refined groups are needed in order to make predictions about how elements will react. The elements are broken up into columns, called groups, and these are defined by the number of electrons in an elements outer shell. The number of electrons in the outer shell controls the chemical properties and the bonding to other atoms, meaning that all elements within a group react similarly. For example, when reacting with water all group 1 metals form a metal hydroxide and hydrogen gas. \\
Elements also belong to a certain period, which is the row in the periodic table that they are in. As the period increases reactions tend to become much more violent. For example, in group 1 Lithium, a period 1 element, gently reacts with water, while caesium, also group 1 but period 6, explodes violently. \\
	\par
	Another pattern shown by the periods is the trend from metal to non-metal, although there are also trends in the elements chemical properties. For example, in forming oxides. \\

	\begin{tabularx}{\textwidth}{c c c c c c c X}
		Na$_2$O & MgO & Al$_2$O$_3$ & SiO$_2$ & P$_2$O$_5$ & SO$_2$ & CLO & Formula of the oxide \\
		0.5 & 1 & 1.5 & 2 & 2.5 & 2 & 1 & Number of oxygens per atom in oxides \\
	\end{tabularx}

\section{Structure of the atom}
	Most of the material in this section was convered in Fundamentals of Materials lectures 1 and 2. \\
	The ionisation energy is the amount of energy required to displace a valence electron from an atom or ion. Using a graph of the Number of ionisations (x-axis) against the log(ionisation energy) (y-axis) we can tell how many shells an atom has and also how many atoms are in each of these shells. We can use known high energy x-rays to find ionisation energy. \\
	
	\subsection{Ionisation Energy}
		The ionisation energy is the amount of energy required to ionise an atom. If an atom has more than one electron then multiple electrons can be removed, however this requires more and more energy as the electrons become closer to the nucleus so have a stronger binding energy. \\
		The below graph shows the First Ionisation Energies of the first 20 elements, which is the energy required to remove the first electron from an atom.
		
		\begin{figure}[!ht]
			\includegraphics[width=\textwidth]{"ionization energies".jpg}
		\end{figure}

		As the size of the element increases the first ionisation energy decreases, as the outer electrons are further away from the nucleus and are shielded from the nucleus by the inner shells.
		
	\subsection{Measuring Ionisation Energy}
		\subsubsection{Xray Photoelectron Spectroscopy (XPES)}
			In this technique the atoms are bombarder with high energy X-rays which can knock electrons out of their orbitals. The electrons acquire a kinetic energy the size of which depends on the binding energy of the electrons. The binding energy of an electron is directly related to its ionisation energy.
			
			\begin{align*}
				B &= E - k \\
				\intertext{Where:} \\
				B &= \text{binding energy of the electron} \\
				E &= \text{the energy of the x-ray} \\
				k &= \text{the kinetic energy aquired by the electron}
			\end{align*}

			
\newpage

\section{Physical/ Inorganic Titration}

	\begin{description}
		\item[O]xidisation
		\item[I]s
		\item[L]oss (of electrons)
		\item
		\item[R]eduction
		\item[I]s
		\item[G]ain (of electrons)
	\end{description}

	The pattern of ionisation energy of each atom repeats each period, although tends lower as you go further down the periodic table.

\newpage

\section{Molecular Forces}
	Intermolecular attractions are attractions between one molecule and a neighbouring molecule. Don't confuse these with intramolecular forces which hold a molecule together, such as covalent bonds. Covalent bonds are very strong while intermolecular forces are weak.
	
	\subsection{Dipoles}
		Electronegativity is a measure of how strongly an atom attracts bonded electrons within a molecule. In general metals have a low value while non-metal have a higher value, so a stronger desire to gain electrons. \\
		A high difference in electronegativity causes the atom with the higher value to pull bound electrons towards it. This causes that atom to gain a slight negative charge, while the atoms that the electrons are being pulled away from have a slight positive charge. For example: \\
		\\
		\chemfig{^{\delta +}H-[:38] \chemabove{\lewis{1:3:,O}}{\quad\scriptstyle \delta +} -[::-76] H^{\delta +}}

	\subsection{Hydrogen Bonding}
		Hydrogen bonding is a type of bonding that occurs in substances where the molecules are made up of hydrogen bonded to a highly electronegative element. This causes a dipole, as we see above in water, and the positively charged hyddrogen from one molecule attracts the negatively charged element, in waters case oxygen, from another molecule. This binds the molecules together with a relatively weak force.

	\subsection{Van der Waals forces}
		Van der Waals forces are weak, short range forces between molecules that are caused by molecules becoming temporary dipoles. These temporary dipoles exist because, at any given instant, the electron cloud will not be perfectly symmetrical. As we move along the groups and down the periods of the periodic table these forces become stronger. \\
		These forces are weaker in small, simple molecules like CH$_4$, but in large molecules with many contact points they are substantial, such as in polymers like polythene. \\
		\par
		This causes a problem for ideal gas laws, as they assume that gas molecules do not exert force on each other. However, at standard pressures and temperatures these forces are not significant and so can be ignored. \\
		\par
		Graphite is carbon allotrope, meaning there are different forms of the same element in the same state (eg graphite, diamond). Within layers there are strong covalent bonds, while between layers there are weak Van der Waals forces. This allows the layers to slide over each other, acting as a solid lubricant. 

	\subsection{Giant Molecular Materials}
		Metallouds form giant structures in their compounds, undergoing covalent bonding to give full outer shells. These bonds are incredibly strong, meaning they have:
		\begin{itemize}
			\item High melting points
			\item Hard and rigid
			\item Some semiconductive properties
			\item Insoluble in all types of solvents
		\end{itemize}


\newpage

\section{Chemical Reactions: Equations and formulae}
	Molecular Mass of a molecule is the sum of the atomic weights. For example, CaCO$_3 = 40 + 12 + (3 * 16) = 100$ gmol$^{-1}$ 
	\[ \text{Number of Moles}(n) = \frac{\text{Mass}(kg)}{\text{Molecular/ Atomic Weight }(mw/aw)} \] 
	So, in 4Kg of limestone (CaCO$_3$) there are $\frac{4000}{100} = 40$ moles. 
	\[ \text{Concentration} = \frac{\text{Number of Moles}(n)}{\text{Volume }(dm^3)} \]
	Note that $dm^3$ is equivalent to a litre so concentration is often given with the units (mol L$^{-1}$). \\
	\par
	Calculate the mass of a product from a known mass of reactant: Fe extraction form its ore be CO, How much iron if we start with 10Kg of ore?\\
	\par
	\ce{Fe_2O_3 + 3CO \rightarrow  \ce{2Fe + 3CO_2}} \\
\\

		\begin{equation*}
		\begin{aligned}[c]
			\text{Molar Ratio} &= \frac{2 \text{mols Fe}}{1 \text{mol Fe$_2$O$_3$}} \\
			n(Fe_2O_3) &= \frac{mass}{mw} \\
			&= \frac{10000}{160} \\
			&= 62.5  \text{ mols of Fe}
		\end{aligned}
		\hfill
		\begin{aligned}[c]
			mw(Fe_2O_3) &= 160 \\
			\text{Using molar ratio} n(Fe) &= 62.5 * \frac{2}{1} \\
			&= 125 mol Fe \\
		\end{aligned}
		\end{equation*}

	$n(Fe) = \frac{mass}{aw}$ so:
	
	\begin{center}
	$mass = n(Fe) * Am(Fe) = 125* 56 = 7000g = 7Kg$
	\end{center}
	
	\subsection{Types of reactions}
		\subsubsection*{Redox Reactions}
			A full equation made from two half reactions, an oxidation reaction and a reduction reaction.
		\subsubsection*{Acid base reaction}
			A reaction between an acid and a base. An acid is a compound that contains hydrogen and reacts with water to form hydronium ions. Strong acids are completely ionised in solutions, while other acids do not completely ionise. \\
			An alkali or base is a substance which produces hydroxide ions in solution. Strong bases ionise completely in solution, while weak bases do not. Metal oxides can also be bases, such as CaO which reacts vigorously with water. Some metal bases are Amphoteric, and react with both acids and bases.
		
		\subsubsection*{Percipitation reaction}
			A reaction between two aqueous substances that form a solid precipitate.
\newpage

\section{Redox reactions and the Electrode potential}
	\subsection{Competition for Electrons}
		If you place a iron rod into a copper sulphate solution the copper plates the iron while the iron dissolves. 
		\[\ce{Fe + Cu^2+ \rightarrow Fe^2+ + Cu}\]
		This reaction involves two half reactions, the oxidation of iron and the reduction of copper.
		\[ \ce{Fe \rightarrow Fe^2+ +2e-} \]
		\[ \ce{Cu^2+ + 2e- \rightarrow Cu} \]

		Iron is a stronger reducing agent than copper while copper ions are stronger oxidising agents than iron ions. This is why this reaction occurs in this way. 
	
	\subsection{Half Cells and Cell Voltages}
		By physically seperating these two half reactions and representing them in seperate pots we now have what is known as two half cells. In each half cell we have a sample of the metal, acting as an electrode, dipped into a solution of its ions. 
		\begin{figure}[!h]
			\centering
			\includegraphics[width=0.5\textwidth]{"Half Cell".png}
		\end{figure}
		
		The two metal electrodes are connected using a wire and volt meter while the two solutions are connected by an ionic conductor salt bridge. Once connected, the iron begins to dissolve as it oxidises and electrons flow from the iron to the copper, where they then reduce the copper ions. The electrode that is undergoing oxidation, in this case iron, is known as the anode while the electrode that's undergoing reduction, in this case copper, is known as the cathode. As electrons are flowing between the two electrodes there must be a potential difference between them, which is known as the electrode potentials. This electrode potential is a direct measure of the ability of a metal, or material, to be oxidised or reduced. In this case the electode potential (E) is 0.78v.
		
	\subsection{Standard Hydrogen Half Cell}
		If different half cells are used then a wide range of potentials are possible. This isn't helpful for comparing the relative strength of reactants, but is the main principle behind modern batteries. \\
		In order to compare different reactants a standard is needed, hence the use of the standard hydrogen half cell. In this half cell the half reaction is the reduction of hydrogen ions to hydrogen gas.
		\[ \ce{2H+ + 2e- \rightarrow H2} \qquad E=0.00v \]
		The standard hydrogen half cell is a plated platinum electrode which is in contanct with 1 mol per dm$^{3}$ acid solution and hydrogen gas at 1atm pressure at a temperature of 298K, which is room temperature. All other cells can then be measured with respect to this cell, keeping the concentration of the solution and the temperature constant. \\
		\begin{figure}[!h]
			\includegraphics[width=\textwidth]{"Reduction Potential".jpg}
		\end{figure}
		\par
		In the table below the standard electrode potentials for many systems are shown, each expressed as reduction reactions. If you wish to use these potentials but the material is being oxidised the sign for the E needs to be change (eg positive to negative). \\
		Consider potassium ions as oxidising agents
		\[ \ce{K+ + e- \rightarrow K} \qquad E = -2.92v \]
		This tells us that is it extremely difficult to convert potassium ions to potassium metal as potassium ions are very poor oxidising agents. If we reverse the reaction we also reverse the sign of the electrode potential
		\[ \ce{K \rightarrow K+ + e-} \qquad E=+2.92v \]
		This tells us that potassium metal is an extremely strong reducing agent and will give up an electron incredibly easily. This means that potassium metal is a very poor choice for structural applications as it will react violently with many substances. 
		
		\pagebreak
		
	\subsection{Cell Potentials}
		Consider again the cell made from the copper electrode and the iron electrode dipped into molar ion solutions. 	If we look at the table then we can see the stronger oxidising agent has reacted with the stronger reducing agent to form a weaker reducing agent and a weaker oxidising agent. A positive cell potential means the reaction will occur, while a negaive potential means the reaction is very unlikely to occur. 
		\[ \ce{Cu^2+ + Fe \rightarrow Cu + Fe^2+} \]
		\[ \ce{Cu^2+ + 2e- \rightarrow Cu} \qquad E = 0.34v \]
		\[ \ce{Fe \rightarrow Fe^2+ + 2e-} \qquad E = 0.44v \]		
		\begin{align*}
			\text{Total Cell Potential} \qquad E &= 0.34 + .044 \\
	       		&= 0.78v
		\end{align*} 
		
		Consider if we replaced the copper cell with a zinc cell
		\[ \ce{Zn^2+ + Fe \rightarrow Zn + Fe^2+} \]
		\[ \ce{Zn^2+ + 2e- \rightarrow Zn} \qquad E = -0.76v \]
		\[ \ce{Fe \rightarrow Fe^2+ + 2e-} \qquad E = 0.44v \]		
		\begin{align*}
			\text{Total Cell Potential} \qquad E &= -0.76 + .044 \\
	       		&= -0.32v
		\end{align*} 			
 			This means that the reaction would actually occur in reverse, with the zinc dissolving and the iron forming. When working out cell potentials do not multiply half cells by the stoichiometry, as it only effects the current flowing, not the potential difference. 
 			
	\subsection{Why We Care}
		\subsubsection*{Predicting the success of a reaction}
			As shown above we can use electrode potential to predict whether a chemical reaction will occur. This is helpful, although it gives no indication of the speed of the reaction.
		\subsubsection*{Corrosion Reactions}
			The anode reaction for a metal is \ce{M \rightarrow M^n+ + ne-} and the cathode reaction for an acidic solution is \ce{2H+ + 2e- \rightarrow H2}, while under normal pH the cathode reaction is \ce{O2 + 2H2O + 4e- \rightarrow 4OH-}. We can use these to predict whether a metal will corrode either in acid or in normal pH levels by finding the Total Cell Potential of the reaction. For example, for copper in acidic solution
			\[ \ce{Cu \rightarrow Cu^2+ + 2e-} \qquad E=-0.34v \]
			\[ \ce{2H+ +2e- \rightarrow H2} \qquad E=0.00v \]
			\begin{align*}
				\text{Total Cell Potential} &= 0.00v -0.34v \\
									&= -0.34v
			\end{align*} 
			So copper will not corrode in acid. In a normal pH
			\[ \ce{Cu \rightarrow Cu^2+ + 2e-} \qquad E=-0.34v \]
			\[ \ce{O2 + 2H2O + 4e- \rightarrow 4OH-} \qquad E=0.40v \]
			\begin{align*}
				\text{Total Cell Potential} &= 0.40v -0.34v \\
									&= 0.06v
			\end{align*} 
			So copper will corrode in a normal pH solution. 
			
		\subsubsection*{Corrosion Protection}
			One of the main ways of protection metals from the damaging effects of corrosion is by galvanising. In this method a more electropositive metal is used in a more electropositive will corrode in preference to the other metal which will act as the cathode. Zinc is often applied as a coating or a sacrificial anode to protect iron from corrosion. 
			
		\subsubsection*{Plating}
			Some metals will plate naturaly onto iron (copper, silver etc) and this can be used as a method of metal recovery as well as plating. Zinc is often plated onto iron, although this will not plate directly from solution since the electrode potential for the reaction is negative. There are two ways to force this reaction; the first is to dip the iron into molten zinc (hot dipped galvanising) and the second is to make the iron very negatively charged. This will attract zinc ions in solution to the iron cathode if the potential of the metal is more negative than zincs reduction potential. This plating mechanism is used in the manufacture of high quality automotive steels. 
			
		\subsubsection*{Extraction}
			Different metals have different ease of extraction. For example, gold is found as a pure metal, copper is easy to extract by heating its sulphide ore, iron requires the use of chemical reducing agen and aluminium can only economically be extracted by electrolysis. This is reflected in the electrode potentials for the different metals
			
			\begin{align*}
				\ce{Au+ + e- \rightarrow Au} \qquad &E = 1.44v \\
				\ce{Cu^2+ + 2e- \rightarrow Cu} \qquad &E = 0.34v \\
				\ce{Fe^2+ + 2e- \rightarrow Fe} \qquad &E =-0.44v \\
				\ce{Al^3+ + 3e- \rightarrow Al} \qquad &E =-1.66v \\
			\end{align*}
			
		\subsubsection*{Electrolysis}
		
\newpage
		
\section{Non-Metallic Elements and Compounds}
	Atoms have two types of bonds, primary and secondary bonds. Primary bonds are strong, short range and include Covalent, Ionic and Metallic bonds, while secondary bonds are weak, long range and include Polar/ Hydrogen bonds and Van der Walls bonds. Primary bonds are rarely pure bonds, and most compounds are partially one type of bonding and partially another type. \\
	\newline
	\begin{tabularx}{\textwidth}{X X}
		\textbf{Pure Covalent} & \textbf{Pure Ionic} \\
		Between atoms of the same group or element & Between group one, two and seven \\
		Very strong, directional bonds & Strong, non-directional bonds \\
		Insoluble in water & Soluble in water \\
		Some for giant structures with very high melting points & High(ish) melting points \\
		Others form isolated molecules with very low melting points & \\
	\end{tabularx}
	
\newpage

\section{Le Chateliers Principle}
	A system will react to changes in its composition in such a way as to try and restore equilibrium. This means if a system is in equilibrium then any change in concentration, temperature or pressure will force a reaction to occur which reduces the effects of this change. This allows us to force systems to react more by tricking the system into not being at equilibrium. \\
	\ce{CaCO3 <=> CaO + CO2} \\
	At 800 \degree C equilibrium rapidly set up. LCP is used to force lime (CaO) production by flushing hot air through to remove \ce{CO2} \\
	\par
	In general: \\
	\ce{A + B <=> C} \\
	Rate of reaction (forwards) = k$_f$[A][B] \\
	Rate of reaction (backwards) = k$_b$[C] \\
	At equilibrium both become equal so \\
	k$_f$[A][B] = k$_b$[C] \\
	\par
	\[ \frac{k_f}{k_b} = \frac{[C]}{[A][B]} = K_c \]
	Where $K_c$ is the equilibrium constant
	
	\subsection{Equilibrium Constant}
		Equilibrium occurs when the rate of forward reaction and the rate of reverse reactions are the same. At this point the concentrations are fixed. $K_c$ is the equilibrium constant and, in the reaction \ce{aA + bB <=> cC + dD}, can be difined as 
		\[ K_c = \frac{[C]^c[D]^d}{[A]^a[B]^b} \]
		A large reaction of $K_c$ means that our reaction is very likely to occur in the direction we want (from left to right) while if $K_c$ is very low the reaction is likely to occur in the other direction.
	
	\subsection{Heterogeneous Systems}
		Most systems involve solids combined with either liquids or gases. In a solution we refer to the material in terms of concentration, however in a solid concentration is meaningless so when calculating $k_c$ we say any solids have a value of one. If a gas is present then partial pressure is usually used instead of $k_c$.
		
	\subsection{Partial Pressure}
		When looking at Partial Pressure of a gas, for example \ce{CO2}, we need to consider $PV=nRT$. In a system with a single gas present $\frac{n_{CO_2}}{V}$ is equivalent to concentraction, so 
		\[ P_{CO_2} = [CO_2] RT \]
		In a mixture of gases at a certain pressure (P) in a fixed volume (V), for example \ce{CO} and \ce{CO2} then the total number of moles of gas is
		\begin{align*}
			n &= n(CO_2) + n(CO) \\
			P &= \frac{[n(CO_2) + n(CO)] RT}{V} \\
			P &= P_{CO_2} + P_{CO}
		\end{align*}
		Within gaseous systems, the equilibrium constant is usually represented in terms of pressure ($K_p$)
		
	\subsection{Measuring K}
		\begin{itemize}
			\item Write down a balanced chemical equation
			\item Write down an equilibrium constant expression
			\item Mix known quantaties of materials together
			\item Wait for concentrations to stabalise, measure them and place them in the expression for K
		\end{itemize}
		
	\subsection{Temperature}
		Changing the temperature can be used to control the reaction. If the temperature is increased then an exothermic reaction will react by shifting the equilibrium to the reactants side, so $K$ may decrease. An endothermic reaction however will react by pushing the equilibrium to the product size, meaning $K$ may increase.
		
	\subsection{Catalytic Impurities} 
		Catalysts increase the rate of reaction, which is speed at whichequilibrium is obtained. As the rate of forward and revers reaction is equally effected the value of $K$ is not affected
		
	\subsection{pH}
		The pH scale can be useful when looking at concentration where the values are too small to comprehend. For example if we looked at the dissociation of water \ce{H2O <=> H+ + OH-} then $K_w = [H^+][OH-] = 10^{-14} \quad mol^2  L^{-2}$. If we have an acidic solution of $[H^+] = 10^{-2} \quad mol  L^{-1}$ so $K_w = 10^{-14} \quad mol^2  L^{-2}$. \\
		pH = $-log_{10}[H^+]$, for example in pure water $[H^+] = [OH^-] = 10^{-7} \quad mol L^{-1}$, therefore pH $= -log_{10}[10^{-7}]=7$. The pH is a scale of concentration compressed to a much smaller range so it is much more easily understood. 
		
	\subsection{Worked Examples}
		\subsubsection*{1.}
			\ce{CaCo3(s) <=> CaO(s) + CO2(g)} \\
			\begin{align*}
				K_c &= \frac{[CaO][CO_2]}{CaCO_3} \\
				&= \frac{1[CO_2]}{1} \\
				&=[CO_2]
			\end{align*}
			To find units for the equilibrium constant we need to perform dimensional analysis. \\
			\begin{align*}
				Units &= \frac{(mol\,L^{-1})^2}{mol\,L^{-1}} \\
				&= mol\,L^-1
			\end{align*}
			so
			\[ K_c = [CO_2] \quad mol\,L^{-1} \]
			
		\subsubsection{2.}
			\ce{2SO2(g) + O2(g) <=> 2SO3(g)} \\
			\begin{align*}
				Units &= \frac{(mol\,L^{-1})^2}{(mol\,L^{-1})^2(mol\,L^{-1})} \\
				&= L\,mol^{-1}
			\end{align*}
			
			\begin{align*}
				K_p &= \frac{P_{SO_3^2}}{P_{SO_3^2} \times P_{O_2}} & K_p &= \frac{[SO_3] \times (RT)^2}{[SO_2]^2[O_2] \times (RT)^3} \\
				P_{SO_3} &= [SO_3]RT & K_c &= \frac{[SO_3]}{[SO_2]^2[O_2]}\\
				P_{SO_2} &= [SO_2]RT & K_p &= K_c \times \frac{(RT)^2}{(RT)^3} \\
				 P_{O_2} &= [O_2] \\
				 \intertext{So}
			\end{align*}
				\[ K_p = \frac{K_c}{RT} \quad L \, mol^{-1} \]

\newpage

\section{Kinetics of Reactions}
	$\Delta$H is the energy absorbed or released during a reaction. \\
	Reactions are likely to be favourable when $\Delta$H is large and negative, so in very exothermic reactions. Thermodynamics predicts these reactions will occur while kinetic dynamics will tell you how fast they occur. \\
	\subsection{Activation Energy}
		If we consider methane
		\ce{ CH4 + 2O2 -> CO2 + 2H2O} \\
		$\Delta H =-802 KJmol^{-1}$ so this reaction is thermodynamically unstable, however if we allow methane to mix with oxygen it will not spontaneously combust, a spark is required to over come the activation energy. This means it is kinetically stable, although once lit the activation energy continues to be supplied by the reaction. \\
		Activation energy is the kinetic barrier to a reactions progress, so a small activation gives a faster process, while a large activation energy can slow or stop a reaction completely. Activation energy can be shown with energy level diagrams.
		
	\subsection{Reaction Rate}
		This is a measure of how fast the reaction is going. It 
		
\newpage

\section{Organic Chemistry}
	Organic chemistry refers to any carbon-based compounds. A hydrocarbon is an organic compound and contains only hydrogen and carbon atoms. There are over seven million known organic compounds, more than all other elements combined.
	
	\subsection{Why so many compounds?}
		There are so many organic compounds because carbon ($1s^2 \, 2s^2 \, 2p^2$, group four) can form four covalent bonds. It can form covalently bonded rings or chains (called catenation) which allows huge flexibility in bonding, and with four or more carbon atoms side chains can form on the main chain, allowing a massive variety of compounds to form. \\
		\par
		Catenation does happen with other elements, but they are much less suited to it. For example, phosphorus (group five) can form chains but dissociatets easily and sulphur (group six) can only form two covalent bonds so chains and rings can't have side groups. \\
		Silicone (group four) should act similarly to carbon but doesn't form many long chain molecules as the covalent bonds in carbon are stronger than those is silicon. 
		
	\subsection{Functional Groups}
		Carbon atoms in an arganic molecule are relatively unreactive, so the chemical reactions and properties are controlled by groups of atoms that are attached to organic compounds. These groups are called functional groups. 
		\subsubsection*{Common Functional Groups}
			\begin{tabularx}{\textwidth}{l l X}
				\textbf{Alkanes} &-&The simplest hydrocarbons \\
				& - & Are present in natural gas, petrols, oils, greases, waxes and polyethene \\
				& - & Can form ringlike structures called Cycloalkanes \\
				& - & Straigh chain molecules are less volatile than branched ones (stronger then Van der Walls bonds) \\
				&&\\
				\textbf{Alkenes} & - & At least one \ce{C=C} \\
				& -&More reactive than alkanes \\
				&-&They are the starting monomer for making polymers \\
				&-& More rigid and more volatile than alkanes as there are weaker Van der Waals bonds \\
				&&\\
				\textbf{Alkynes} & -& At least one \ce{C#C} \\
				&-&Even more reactive but less widely used \\
				&-&Very rigid, non-soluble in water and tends to have a slightly higher boiling point than alkanes and alkenes \\
				&&\\
				\textbf{Benzene}&-&Compunds containing a benzene ring are called aromatic compounds \\
				&-&Used in a lot of things as very important building block \\
				&-&Not very reactive \\
				&-& Aromatic compounds are often a health hazard \\
				&&\\
				\textbf{Halogens} &-&Quite a few uses in industry \\
				&-& Low flammability so useful in fire retardant, refrigeration, solvents etc. \\
				&-& \ce{CHCl3} was previously used as an anaesthetic \\
				&-& Suspected of being environmental pollutants and health hazards \\
			\end{tabularx}		
		
	

\end{document}