\documentclass[a4paper, 12pt]{article}

\usepackage{graphics}
\usepackage{graphicx}
\usepackage{mhchem}
\usepackage{pgf}
\usepackage{chemfig}
\usepackage{sidecap}
\usepackage{gensymb}
\usepackage{tabularx}

\begin{document}
	\title{Fundamentals of Materials (EG-080)}
	\date{}
	\maketitle
	
	\section*{Introduction}
		Lecturer Dr Shirin Alexander, available in room 204 ESRI building. Office hours are Friday 1-3pm, although she needs to be emailed before you turn up. 

	\subsection*{Recommended Reading}
		\begin{itemize}
			\item Chemistry : molecules, matter, and change / Peter Atkins, Loretta Jones.
			\item General Chemistry E2 *Free*Aie by BELLAMA
			\item Engineering materials 1 : an introduction to properties, applications and design / Michael F. Ashby and David R. H. Jones.
			\item Materials science and engineering : an introduction / William D. Callister, Jr.
		\end{itemize}
	
	\subsection*{Testing}
		There will be three Blackboard tests, worth a total of 25\%, each test being worth 8\%, 9\%, and 8\% respectively. There will also be a final exam in January worth 75\%
	
	\newpage		
	\section{Atoms and Elements \hfill \date{03/10/2017}}
		\begin{description}
			\item [Matter -] Has volume and mass 
			\item [Substance -] A pure form of matter, containing only a single type. For example, pure water 
			\item [Element -] A substance composed of a single kind of atom 
			\item [Isotope -] An atom with the same atomic number but a different molecular weight
			\item [Homogenous Mixture -] A solution that will naturally seperated if left 
			\item [Heterogeneous Mixtures -] A mixture of substances that require a physical technique to seperate
		\end{description}
		
		\subsection*{Atomic Configuration}
			Proton Mass (Positive) = $1.67\times10^{-24}$ \\
			Neutron Mass (Neutral) = $1.67\times10^{-24}$ \\
			Electrons (Negative) = $9.11\times10^{-28}$ \\
			
			\ce{^{A}_{Z}X} \\
			Where: \begin{description}
				\item [Z - ] Is the atomic number, the number of protons
				\item [A - ] Is the atomic weight, the total number of protons and neutrons
				\item [X - ] Is the atomic symol
			\end{description} 
			
	
			
			Quantum numers for an atoms electrons can be used to calculate properties of the electrons, such as their energy
			\begin{description}
				\item [n - ] Principal quantum number (or Shell number) represents the energy of the electron. The greater n, the higher the shells energy level and the weaker it's bound to the nucleus
				\item [l - ] Orbital angular quantum number. This specifies the shape of the orbital
				\item [m$_l$ - ] Magnetic quantum number. This specifies the indivdual orbital of a particular shape and is also associated with the orbital direction
			\end{description}
			
			The number of electrons that an occupy a single shell is $2n^2$ 		

			\begin{figure}[t]
				\includegraphics[width=\textwidth]{"Electron orbitals".jpg} 
				\caption{The different possibilities for s, p, and d orbitals}
			\end{figure}
			
		\subsection*{Implications of Quantum Physics}
			\begin{itemize}
				\item Electrons can only occupy discrete orbitals
				\item Orbitals have different energies, shapes and directions
				\item There are only a maximum of 2 electrons per orbital (spinning opposite ways)
				\item Electrons will fill empty shells first before doubling up
				\item Orbitals are clouds of probablility, not true orbits
				\item Orbitals can be represented using a number for the energy level and a letter for the shape. For example 1s, 2p, 3d 
			\end{itemize}
			
			Electrons in electron shells can be represented using written notation. For example, Mg $= 1S^2 2S^2 2p^6 3S^2$. This can be written shorthand by using the previous nobel gas to represent complete shells. For example, Mg $=[Ne] 3S^2$
			\newpage
			\clearpage
			
			\begin{figure}[!ht]
				\includegraphics[width=\textwidth]{"Energy levels of orbitals".png}
				\caption{}
			\end{figure}

			\begin{figure}[!ht]
				\includegraphics[width=\textwidth]{"Filling orbitals".png}
				\caption{}
			\end{figure}
			
			\begin{figure}[!ht]
				\includegraphics[width=\textwidth]{"Orbital diagram".png}
				\caption{This diagram can be used to show how electrons fill shells. Within each row electrons will always g to an empty box before filling a box completely. These boxes represent the different options withing each shell layer (reference Figure 1)}
			\end{figure} 
			
			\begin{figure}[!ht]
				\includegraphics[width=\textwidth]{"Exceptions".png}
				\caption{These are the exceptions to these rules. They may ask about the highlighted ones}
			\end{figure}
			
			\clearpage
			\newpage
			
	\section{Atomic Bonding \hfill \date{10/10/17} }
		The Valence Electrons (VE) of an atom are the electrons in the outer shell of an atom, so are the ones involved in forming bonds to adjacent atoms. Lewis symbols and be used to show the valence electrons in an atom. For example, \lewis{0.2.4.6.,C} \\
		Atoms aim to gain stability by gaining or losing electrons with an aim to gain the same electronic configuration of the closest (in terms of atomic number) noble gas. The interaction of atoms through chemical bonding leads to an overall decrease in the energy. The energy is stored as potential energy. \\
		\par
		Ionic and covalent are terms used to describe two extremes of chemical bonds. In most substances the bonds lie between purely covalent or covalent. When describing bonds between non-metals covalent bonding is a good metal, while when the bond is between metals and non-metals ionic bonding is a good model.
		\subsection*{Ionic Bonding}
			Ionic bonding occurs when one atom looses at least one electron from its valence band to another atom to gain stability. Atoms that undergo ionic bonding become positively/ negatively charged.
			
			\subsubsection*{Salt, NaCl}

				\begin{tabular}{l l}
				 	$ \text{Na} = 1s^2, 2s^2, 2p^6, 3s^1 $ & \lewis{0.,Na} \\
				 	$ \text{Cl} = 1s^2, 2s^2, 2p^6, 3s^2, 3p^5 $ &  \lewis{0:2:4.6:,Cl} \\
				\end{tabular}
				
				For Na and Cl the third shell holds the valence electrons. Na loses one electron while Cl gains it, producing Na$^{^+}$ \lewis{0:2:4:6:,Cl}$^{^-}$ \\
				\par
				Ionic solids tend to stack together in regular crystalline structures as the charged electrons in the molecule atract to the other charged atoms in other molecules. This strong electrostatic attraction between oppositely charged ions in ionic solids accounts of their typical properties such as high melting and boiling points, as well as brittleness. \\
				When an ionic solid is hit the positive ions that normally line up with negative ions now line up with positive ions, forcing the lattice apart. This is why ionic solids are brittle. \\ 
				In the lattice strong coulomb forces ionically bond each Na$^+$ ion to six neighbouring Cl$^-$, meaning it takes a lot of energy to break all of these bonds. This accounts for the high melting and boiling points. 
				\par
				The ionisation energy of an atom increases as its position in the periodic table moves down and to the right. This is based on the distance of the nucleus to the valence shell.
				
		\subsection*{Covalent Bonding}
			Covalent bonding occurs when pairs of valence electrons are shared between two atoms. This means that neither atom gains or loses any charge. Non-metal elements tend to interact via covalent bonding (instead of ionic) because their ionisation energy is too high. \\
			Covalent bonds are very strong; it takes a large amount of energy to seperate atoms bonded in this way. For example, diamond is 100\% covalent bonds, while graphite is only partially covalent.
			\subsubsection*{Hydrogen}
				\begin{tabular}{l l l}
					$\text{H} = 1s^2 $ & \lewis{0.,H} & \lewis{4.,H} \\
				\end{tabular}
				\newline
				Hydrogen wants to be as stable as the closest noble gas, Helium. The sum of attractive electrostatic forces (nucleus $\leftrightarrow$ electrons) and repulsive electrostatic forces (electron $\leftrightarrow$  electron, nucleus $\leftrightarrow$ nucleus) it overall attractive therefore bonding occurs. \\
					\begin{center}
						\lewis{0.,H} \space \space \lewis{4.,H} $\rightarrow$ \lewis{0:,H} H $\rightarrow$ \chemfig{H - H}
					\end{center}
				
				Here the two hydrogen atoms share their bond, resulting in a single covalent bond being formed.
			
				
			\subsubsection*{Water}
			
				\begin{tabular}{l l}
					$\text{H} = 1s^2 $ & \lewis{0.,H} \\
					$\text{O} = 1s^2, 2s^2, 2p^4$ & \lewis{0:2.4:6.,O} \\	
				\end{tabular}
				
				\begin{center}
					\lewis{0.,H} \space \space \lewis{0.2:4.6:,O} \space \space \lewis{4.,H} $\rightarrow$ H \lewis{0:2:4:6:,O} H $\equiv$ \chemfig{H - \lewis{2:6:,O} - H}
				\end{center}
					
				The presence of the doublet of negative electrons the two hydrogens are squeezed together as the two doublets repel from each other. This force results in water having a bond angle of 104.5\degree. The oxygen atom pulls on the shared electrons in each hydrogen, meaning one side of the molecule is positively charged and the other side is negatively charged. This means it is a polar molecule. \\
				In liquid water these partial charges cause weak electrostatic attraction between molecules, called hydrogen bonds, that keep the molecules together.
				 
		\subsection*{Electronegativity}
		
		\begin{figure}[!ht]
			\includegraphics[width = \textwidth]{"Electronegativity".png}
		\end{figure}
			Electronegativity defines the power of an atom in a bond to draw electrons from the other atom.  We can use the difference in electronegativity between two elements to predict the most appropriate bonding model for a chemical bond between them. In general, if the electronegativity difference, x, between the two elements is: 
			\[x \geq 1.7 \rightarrow \text{ionic bond} \]
			\[x < 0.5 \rightarrow \text{covalent bond} \]
			\[0.5 \leq x < 1.7 \rightarrow \text{polar covalent bond} \]
		
		\newpage
				
	\section{Metallic Bonding \hfill \date{17/10/17}}
	
		\begin{description}
			\item[Delocalisation - ] When an electron is freed from its orbital
			\item[Malleable - ] Can be pressed to form sheets
			\item[Ductile - ] Can be stretched to form wires
		\end{description}
		
		Metallic bonding can be explained according to the Drude model, which is a simplistic model. It was developed to explain the transport of electrons in metals, although does not explain all of their electronic properties. \\
		\par
		The drude model assumes a 'sea of free, vibrating, electrons' which are rebounding off of heavier, relatively immobile ions. Valence electrons are free to drift through the entire metal as they are not bound to any single atom, while the nonvalence electrons and the atomic nuclei form the ion cores. \\
		When metallic atoms come close the valence electrons become delocalised as orbitals overlap and lose energy. This loweres the overall energy and holds the atoms together, forming the sea of free electrons, or "Drude Glue". 
		
		\subsection*{Metallic properties within the Drude model}
			\subsubsection*{Partially explains melting points}
				The overall energy is lowered significantly when the atoms are close together so a lot of energy needs to be added to seperate them. The electrons are homogeneously shared, meaing there is a strong bonding between metal atoms. This gives metals their solid structure. \\
				The more outer electrons an atom has the more "Drude Glue" there is between atoms, meaning there is a higher melting points, although this is only true to a certain extent. Some examples:
				\begin{itemize}
					\item Sodium (1s$^2$, 2s$^2$, 2p$^6$, 3s$^1$ ): 98\degree C 
					\item Magnesium ([Ne] 3s$^2$): 650\degree C
					\item Aluminium ([Ne] 3s$^2$, 3p$^1$): 660\degree C
				\end{itemize}
				
				There are exeptions to this, such as mercury ([Xe] 4f$^{14}$, 5d$^{10}$, 6s$^2$) which has very weak atomic interaction, so very little Drude Glue.
				
			\subsubsection*{Explains conductivity}
				The outer electrons are free to move so can carry electric charge (current). Also, the more valence electrons there are the higher the conductivity. Valence electrons also carry heat through vibrations, so metals are good thermal conductors. 
				
			\subsubsection*{Explains shininess}
				When a photon of light hits a metal surface, knocks an electron up to one of the many empty orbitals of a higher energy. This electron is unstable so drops back down, re-emitting the energy as a photon identical to the first. This is called reflection. 
				
			\subsubsection*{Explains malleabitily}
				Metallic bonding is non-directional, so atoms can move along slip planes by a small a amount. As long as the atoms stay close together the metallic bonds stay in place, meaning metals can be deformed.
				
		
		\newpage
		
	\section{Introduction To Metals \hfill \date{19/10/17}}
		\begin{description}
			\item[Smelting - ] A process where metals are extracted from an ore by using a combination of heat and a reducing agent. 
			\item[Monoatomic - ] A stable molecule composed of a single atom (such as nobel gasses)
		\end{description}
		
		\subsection*{Crystalline vs Amorphous}
		
			\begin{itemize}
				\item Atoms in a crystalline material are in a repeating/ orderly array over large atomic distances.
				\item All metals (and many ceramics) form crystalline structures
				\item Amorphous materials are characterised by no/ very little ordering of their atoms. Their layout is random
			\end{itemize}
			
			Crystalline materials have a small energy difference between the conduction band and the valence band, while amorphous materials have a large energy diference. This is why amorphous materials are insulators, while crystaline materials are conductors or semiconductors.
			
			\subsection*{How atoms in a metal stack}
				There are three main ways in which atoms stack -
				\begin{itemize}
					\item Hexagonal close packing (hcp)
					\item Cubic close/ face-centered cubic packing (ccp/fcc)
					\item Body-centered cubic packing (bcc)
				\end{itemize}
				
				This stacking controls the ductility, electronic and magnetic properties of metals.
				
				\begin{description}
					\item[Coordination Number - ] The number of near neighbours, at equal distance, to a central atom in a crystal				
					\item[Packing Density - ] The fraction of space filled by the atoms (assuming the atoms are perfect spheres)
					\item[Unit Cell - ] The smallest hypothetical unit that when stacked together repeatedly with no gaps will produce an entire crystal
				\end{description}
				
				\subsubsection*{Hexagonal Close Packing (HCP)}
					
					\begin{figure}[!ht]
						\includegraphics{"HCP".png}
					\end{figure}
					
					\begin{itemize}
						\item Each atom is surrounded by 6 others in each layer
						\item Coordination number of 12
						\item 74 \% packing density
						\item 3 slip systems
						\item Examples - Magnesium, Zinc
					\end{itemize}

				\subsubsection*{Cubic Close Packing (CCP)}
					\begin{figure}[!ht]	
						\includegraphics{"FCC".png}
					\end{figure}
					
					\begin{itemize}
						\item Each atom is surrounded by 6 others in each layer
						\item Coordination number of 12
						\item 74 \% packing density
						\item 12 slip systems (4 slip planes, 3 directions)
						\item Examples - Aluminium, Copper, Gold
					\end{itemize}
					
				\subsubsection*{Body Centred Cubic (BCC)}
					\begin{figure}[!ht]
						\includegraphics{"BCC".png}
					\end{figure}
					
					\begin{itemize} 
						\item More open structure
						\item Coordination number of 8
						\item 68 \% packing density
						\item 48 slip systems (but planes are not closely packed, so more force is required)
						\item Examples - Iron, Sodium, Potassium
					\end{itemize}
					
		\subsection*{Slip in metals}
			Metals deform in planes, with one plane of atoms sliding over another plane.  Atoms want to stay as close to eachother as possible, meaning slip is easiest on closely packed planes in closely packed directions. \\
			A slip system describes the set of symmetrically identical slip planes and the associated family of slip directions for which dislocation motion can easily occur and lead to plastic deformation. 
			\subsubsection*{HCP}
				HCP has only one close packed plane, with three directions, so three slip systems. This means it is limited to slipping only if force is applied in certain directions. In most cases, it is quite brittle. 
			
			\subsubsection*{CCP/ FCC}
				CCP has four close packed planes, each with three close packed directions, so 12 slip planes. It can slip in many arrangements, meaning it is ductile. 
				
			\subsubsection*{BCC}
				BCC has the most slip systems (up to 48) but the planes are not tightly packed, meaning a higher force or higher temperature is needed. 
				\begin{figure}[!h]
					\includegraphics{"Plane Slip".png}
				\end{figure}

		\subsection*{Plastic vs Elastic Deformation}
		
			\begin{figure}[!ht]
				\includegraphics{"Deformation".png}
			\end{figure}
			
			Elastic deformation is one is which stress and strain are directly proportional. The deformation is also reversible. The equation for elastic deformation is 
			
			\begin{align*}
				\sigma &= E \epsilon \\			
				\intertext{Where:} 			
			 	\sigma &= \text{Stress} \\
				E &= \text{Young's modulus} \\
				\epsilon &= \text{Strain} \\
			\end{align*}
			
			Plastic deformation is one in which stress and strain are no longer proportional. This deformation is not reversible.
			
			\begin{figure}[!h]
				\includegraphics[width = \textwidth]{"Plastic Deformation".png}
			\end{figure}
			
		\subsection*{Ductile vs Brittle}
			\begin{description}
				\item[Ductility - ] A measure of the extent of plastic deformation a material can sustain before fracture; a ductile metal can be stretched into a wire without breaking
				\item[Malleability - ] The ability of a solid to deform under pressure
				\item[Brittle - ] A material is brittle is it fractures at less than $\sim$5\%strain
			\end{description}
			

			\begin{figure}
				\includegraphics[width = 0.40\textwidth]{"Ductile vs Brittle".png}
			\end{figure}

			Brittle metals have very clean breaks when they fracture, while malleable metals will deform and tear before they break. \\
			Metals can change from ductile to brittle, for example below 912\degree C Iron and steel transform from CCP to BCC, from ductile to brittle. \\
		
	\pagebreak
	
	\section{Properties of metals}
		\subsection*{Metallic Grains}
			As a metal starts to solidify crystals start growing in many different directions. These crystals merge, giving the metal a grain structure. Every grain has its own crystalline orientation.
			
			\begin{SCfigure}
				\caption{Orientation map of an Inconel 600 superalloy sample (Ni- based corrosion resistant alloy)}
				\includegraphics[width=0.48\textwidth]{"Grain Boundaries".png}
			\end{SCfigure}
		
			Slip is difficult across a grain boundary. Smaller grains means more boundaries, reducing slip so forming a harder/ stronger metal. Small grains can be obtained by fast cooling, use of a chemical agent (such as a very strong acid), or through phhysical treatment, like ultrasounds. \\
			Reducing slip in a metal makes the metal harder, normally stronger, but sometimes more brittle. There are two ways to make it harder for atoms to slip, these are:
			\begin{itemize}
				\item Alloying
				\item Decreasing the grain size
			\end{itemize}
				
				\subsubsection*{Alloying}
					\textbf{Alloy - } An alloy is a metallic substance which is composed of two or more elements with a metallic structure. 
					\par
					A mix of differently sized atoms makes it harder for atoms to slip over each other, the the more alloying the harder it is for atoms to slip. For example: \\
					\par
					\begin{tabularx}{\linewidth}{X X}
						Mild steel -0.15\% Carbon & Cast iron - 4\% Carbon \\
						Malleable and ductile & Much harder and brittle \\
						Used for rolled structural sections (tubes, round bars etc.) & Used in foundaries to make complex objects \\
					\end{tabularx}
			
	\newpage
	
	\section{Electrochemical Potential \hfill \date{31/10/17}}
		Electrochemical potential is an alternative way of measuring the oxidising/ reducing power of elements. It is a measure (J/ mol) of the chemical potential of a compound which takes electrostatic forces into account. \\
		It can be used to understand; the use of electricity to extract metals, corrosion, batteries, and fuel cells. It also provides another way to predict the occurremce of redox chemical reactions. 
		
		\subsection*{Standard potential}
			The standard potential is an electrochemical potential which measure the ability of compounds to oxidize/ reduce other compounds in comparison to the H$^+$/ H$_2$ redox couple under standard conditions. Standard potential (E$^0$) is in volts and the sign and magnitude is is an indication of a substance to act as a reducing or oxidizing agent. \\
			\ce{2H^+ + 2e^- \rightarrow H_2(g)} = 0V \\
			
			\begin{figure}[!h]
				\includegraphics[width=\textwidth]{"Standard Potential".png}
			\end{figure}
			\pagebreak
			By convention, the standard potential is associated to a reduction reaction: \ce{M^{n+} + ne^- \rightarrow M} where M represents a metal. E$^0$ = E$^0_{red(uction)}$. This equation can be shown as E$^0_{red}$(M$^{n+}$/M) = Standard potential (V). For example: \\
			\ce{E^0_{red}(Cu^{2+}/Cu) = +0.34V} \\
			\ce{E^0_{red}(Zn^{2+}/Zn) = -0.76V} \\
			\par
			If E$^0_{red}$ > 0 Then M$^{n+}$ tends to be reduced by H$_2$(g): \\
			\ce{M^{n+}(aq) + H_2(g) \rightarrow M(s) + 2H^+(aq)} \\
			\par
			If E$^0_{red}$ < 0 Then M$^{n+}$ tends to reduce H$^+$(aq): \\
			\ce{M(s) + 2H^+(aq) \rightarrow M^{n+}(aq) + H_2(g)} \\
			\par
			For example: \\
			\ce{Cu^{2+} + H_2 \rightarrow H^+ + Cu} \\
			\ce{Zn^{2+} + H^+ \rightarrow Zn + H_2} \\
			
		\subsection*{Group Metals}
			\subsubsection*{Group 1}
				Very reactive, strong reducing agent, and soft with low melting points
			\subsubsection*{Group 2}
				Less reactive than group 1, weaker reducing agents, they are harder and have higher melting points. 
			\subsubsection*{Group 3}
				Weaker reducing agents than group 2 and not very reactive. They are harder and have higher melting points (due to more "drude glue")
			
			\subsection*{Transition Metals}
				Transition metals are metals with partially filled d-orbitals. This controls many properties of transition metals and make the elements all behave similarly. They generally engage in strong bonding as atoms are small and heavy. They are also not very reactive due to their valence electrons having a high ionization energy. They have more than one oxidation state and often show catalytic activity.
				
	\newpage				
\end{document}