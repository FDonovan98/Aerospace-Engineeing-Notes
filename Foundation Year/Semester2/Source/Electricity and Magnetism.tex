\documentclass[a4paper, 12pt]{article}

\usepackage{graphicx}
\usepackage{amsmath}
\usepackage{tabularx}

\begin{document}
\title{Electricity and Magnetism}
\date{}
\maketitle

\tableofcontents

\newpage

\section{Resistance}

	\begin{align*}
		R &= \frac{\rho l}{A}
		\intertext{where:}
		R &= \text{Resistance}\\
		\rho &= \text{Resistivity}\\
		l &= \text{Length}\\
		A &= \text{Area}
	\end{align*}
	
	\begin{align*}
		\rho &= \frac{1}{\sigma}
		\intertext{Where:}
		\rho &= \text{Resistivity} \\
		\sigma &= \text{Conductivity}
	\end{align*}

	\subsection{Power}
		Power is the energy dissipated per second, usually in the form of heat. The unit Watts (W) is used. The power , P, dissipated in a resistor, R, is given by
		\begin{align*}
			P&=IV
			\intertext{Using Ohm's law}
			P&=I^2R\\
			P&=\frac{V^2}{R}
		\end{align*}
	

\end{document}